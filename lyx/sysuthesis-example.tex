%% LyX 2.3.6 created this file.  For more info, see http://www.lyx.org/.
%% Do not edit unless you really know what you are doing.
\documentclass[doctor]{sysuthesis}
\usepackage[T1]{fontenc}
\setcounter{secnumdepth}{3}
\setcounter{tocdepth}{3}
\usepackage{amsmath}
\usepackage{graphicx}
\usepackage{esint}
\usepackage{nomencl}
% the following is useful when we have the old nomencl.sty package
\providecommand{\printnomenclature}{\printglossary}
\providecommand{\makenomenclature}{\makeglossary}
\makenomenclature
\usepackage[unicode=true]
 {hyperref}

\makeatletter

%%%%%%%%%%%%%%%%%%%%%%%%%%%%%% LyX specific LaTeX commands.
%% Because html converters don't know tabularnewline
\providecommand{\tabularnewline}{\\}

%%%%%%%%%%%%%%%%%%%%%%%%%%%%%% User specified LaTeX commands.
% !TeX program = xelatex
% !TeX spellcheck = en_US

\input{sysusetup}

\sysusetup{
    keywords = {关键词 1, 关键词 2, 关键词 3, 关键词 4, 关键词 5},
	keywords* = {keyword 1, keyword 2, keyword 3, keyword 4, keyword 5},  
}

\makeatother

\def\eqdeclaration#1{, see equation\nobreakspace(#1)}
\def\pagedeclaration#1{, page\nobreakspace#1}
\def\nomname{术语}

\begin{document}
\frontcoverpage[file=scan/scan-frontcover.pdf]

\maketitle

\copyrightpage[file=scan/scan-copyright.pdf]

\frontmatter{}
\begin{abstract}
论文的摘要是对论文研究内容和成果的高度概括。 摘要应对论文所研究的问题及其研究目的进行描述,对研究方法和过程进行简单介绍,对研究成果和所得结论进行概括。
摘要应具有独立性和自明性,其内容应包含与论文全文同等量的主要信息。 使读者即使不阅读全文,通过摘要就能了解论文的的总体内容和主要成果。

论文摘要的书写应力求精确、简明。 切忌写成对论文书写内容进行提要的形式,尤其要避免“第 1 章……;第 2 章……;……”这种或类似的陈述方式。

关键词是为了文献标引工作、用以表示全文主要内容信息的单词或术语。 关键词不超过 5 个,每个关键词中间用分号分隔。
\end{abstract}
\begin{abstract*}
An abstract of a dissertation is a summary and extraction of research
work and contributions. Included in an abstract should be description
of research topic and research objective, brief introduction to methodology
and research process, and summary of conclusion and contributions
of the research. An abstract should be characterized by independence
and clarity and carry identical information with the dissertation.
It should be such that the general idea and major contributions of
the dissertation are conveyed without reading the dissertation.

An abstract should be concise and to the point. It is a misunderstanding
to make an abstract an outline of the dissertation and words “the
first chapter”, “the second chapter” and the like should be avoided
in the abstract.

Keywords are terms used in a dissertation for indexing, reflecting
core information of the dissertation. An abstract may contain a maximum
of 5 keywords, with semi-colons used in between to separate one another.
\end{abstract*}
\tableofcontents{}

\listoffigures

\listoftables

\settowidth{\nomlabelwidth}{TIFF }
\printnomenclature{}

\mainmatter{}

\chapter{论文主要部分的写法}

研究生学位论文撰写,除表达形式上需要符合一定的格式要求外,内容方面上也要遵循一些共性原则。

通常研究生学位论文只能有一个主题(不能是几块工作拼凑在一起),该主题应针对某学科领域中的一个具体问题展开深入、系统的研究,并得出有价值的研究结论。
学位论文的研究主题切忌过大,例如,“中国国有企业改制问题研究”这样的研究主题过大,因为“国企改制”涉及的问题范围太广,很难在一本研究生学位论文中完全研究透彻。

\section{论文的语言及表述}

除国际研究生外,学位论文一律须用汉语书写。 学位论文应当用规范汉字进行撰写,除古汉语研究中涉及的古文字和参考文献中引用的外文文献之外,均采用简体汉字撰写。

国际研究生一般应以中文或英文书写学位论文,格式要求同上。 论文须用中文封面。

研究生学位论文是学术作品,因此其表述要严谨简明,重点突出,专业常识应简写或不写,做到立论正确、数据可靠、说明透彻、推理严谨、文字凝练、层次分明,避免使用文学性质的或带感情色彩的非学术性语言。

论文中如出现一个非通用性的新名词、新术语或新概念,需随即解释清楚。

\section{论文题目的写法}

论文题目应简明扼要地反映论文工作的主要内容,力求精炼、准确,切忌笼统。 论文题目是对研究对象的准确、具体描述,一般要在一定程度上体现研究结论,因此,论文题目不仅应告诉读者这本论文研究了什么问题,更要告诉读者这个研究得出的结论。
例如:“在事实与虚构之间:梅乐、卡彭特、沃尔夫的新闻观”就比“三个美国作家的新闻观研究”更专业、更准确。

\section{摘要的写法}

论文摘要是对论文研究内容的高度概括,应具有独立性和自含性,即应是 一篇简短但意义完整的文章。 通过阅读论文摘要,读者应该能够对论文的研究
方法及结论有一个整体性的了解,因此摘要的写法应力求精确简明。 论文摘要 应包括对问题及研究目的的描述、对使用的方法和研究过程进行的简要介绍、
对研究结论的高度凝练等,重点是结果和结论。

论文摘要切忌写成全文的提纲,尤其要避免“第 1 章……;第 2 章……;……”这样的陈述方式。

\section{引言的写法}

一篇学位论文的引言大致包含如下几个部分: 1、问题的提出; 2、选题背 景及意义; 3、文献综述; 4、研究方法; 5、论文结构安排。
\begin{itemize}
\item 问题的提出:要清晰地阐述所要研究的问题“是什么”。\footnote{选题时切记要有“问题意识”,不要选不是问题的问题来研究。}
\item 选题背景及意义:论述清楚为什么选择这个题目来研究,即阐述该研究对学科发展的贡献、对国计民生的理论与现实意义等。
\item 文献综述:对本研究主题范围内的文献进行详尽的综合述评,“述”的同时一定要有“评”,指出现有研究状态,仍存在哪些尚待解决的问题,讲出自己的研究有哪些探索性内容。
\item 研究方法:讲清论文所使用的学术研究方法。
\item 论文结构安排:介绍本论文的写作结构安排。
\end{itemize}

\section{正文的写法}

本部分是论文作者的研究内容,不能将他人研究成果不加区分地掺和进来。 已经在引言的文献综述部分讲过的内容,这里不需要再重复。 各章之间要存在有机联系,符合逻辑顺序。

\section{结论的写法}

结论是对论文主要研究结果、论点的提炼与概括,应精炼、准确、完整,使读者看后能全面了解论文的意义、目的和工作内容。 结论是最终的、总体的结论,不是正文各章小结的简单重复。
结论应包括论文的核心观点,主要阐述作者的创造性工作及所取得的研究成果在本领域中的地位、作用和意义,交代研究工作的局限,提出未来工作的意见或建议。
同时,要严格区分自己取得的成果与指导教师及他人的学术成果。

在评价自己的研究工作成果时,要实事求是,除非有足够的证据表明自己的研究是“首次”、“领先”、“填补空白”的,否则应避免使用这些或类似词语。

\chapter{图表示例}

\section{插图}

图片通常在 \env{figure} 环境中使用 \cs{includegraphics} 插入,如图\textasciitilde\ref{fig:example}
的源代码。 建议矢量图片使用 PDF\nomenclature{PDF}{可携带文档格式(Portable Document Format)}
格式,比如数据可视化的绘图; 照片应使用 JPG 格式; 其他的栅格图应使用无损的 PNG\nomenclature{PNG}{便携式网络图形(Portable Network Graphics)}
格式。注意,\LaTeX 不支持 TIFF \nomenclature{TIFF }{标签图像文件格式(Tag Image File Format)}格式;EPS
格式已经过时。

\begin{figure}
\begin{centering}
\includegraphics[width=0.5\linewidth]{figures/example-image-a}
\par\end{centering}
国外的期刊习惯将图表的标题和说明文字写成一段,需要改写为标题只含图表的名称,其他说明文字以注释方式写在图表下方,或者写在正文中。

\caption{\label{fig:example}示例图片标题}

\end{figure}

若图或表中有附注,采用英文小写字母顺序编号,附注写在图或表的下方。 国外的期刊习惯将图表的标题和说明文字写成一段,需要改写为标题只含图表的名称,其他说明文字以注释方式写在图表下方,或者写在正文中。

如果一个图由两个或两个以上分图组成时,各分图分别以 (a)、(b)、(c)...... 作为图序,并须有分图题。推荐使用\pkg{subcaption}宏包来处理,
比如图\textasciitilde\ref{fig:subfig-a}和图\textasciitilde\ref{fig:subfig-b}。

\begin{figure}
	\centering
	\subcaptionbox{分图 A\label{fig:subfig-a}}{\includegraphics[width=0.35\linewidth]{example-image-a.pdf}}
	\subcaptionbox{分图 B\label{fig:subfig-b}}{\includegraphics[width=0.35\linewidth]{example-image-b.pdf}}   
	\caption{多个分图的示例}
	\label{fig:multi-image}
\end{figure} 

\section{表格}

表应具有自明性。为使表格简洁易读,尽可能采用三线表,如表\textasciitilde\ref{tab:three-line}。
三条线可以使用 \pkg{booktabs}宏包提供的命令生成。

\begin{table}

\caption{\label{tab:three-line}三线表示例}

\centering{}%
\begin{tabular}{ll}
\hline 
文件名 & 描述\tabularnewline
\hline 
thuthesis.dtx & 模板的源文件,包括文档和注释\tabularnewline
thuthesis.cls & 模板文件\tabularnewline
thuthesis-{*}.bst & BibTeX 参考文献表样式文件\tabularnewline
\hline 
\end{tabular}
\end{table}

表格如果有附注,尤其是需要在表格中进行标注时,可以使用\pkg{threeparttable}宏包。研究生要求使用英文小写字母 a、b、c……顺序编号,本科生使用圈码①、②、③……编号。

\begin{table}
	\centering
	\begin{threeparttable}[c]
		\caption{带附注的表格示例}
		\label{tab:three-part-table}
		\begin{tabular}{ll}
			\toprule       
			文件名                   & 描述                         \\       
			\midrule       
			thuthesis.dtx\tnote{a}  & 模板的源文件,包括文档和注释 \\       
			thuthesis.cls\tnote{b}  & 模板文件                     \\       
			thuthesis-*.bst        & BibTeX 参考文献表样式文件    \\       
			\bottomrule     
		\end{tabular}     
		\begin{tablenotes}       
			\item [a] 可以通过 xelatex 编译生成模板的使用说明文档;使用 xetex 编译 \file{thuthesis.ins} 时则会从 \file{.dtx} 中去除掉文档和注释,得到精简的\file{.cls} 文件。       
			\item [b] 更新模板时,一定要记得编译生成 \file{.cls} 文件,否则编译论文时载入的依然是旧版的模板。     
		\end{tablenotes}   
	\end{threeparttable} 
\end{table}

如某个表需要转页接排,可以使用\pkg{longtable}宏包,需要在随后的各页上重复表的编号。编号后跟表题(可省略)和“(续)”,置于表上方。续表均应重复表头。

\begin{longtable}{cccc}     
	\caption{跨页长表格的表题} \\     
	\toprule     
	表头 1 & 表头 2 & 表头 3 & 表头 4 \\     
	\midrule   
	\endfirsthead     
		\caption[]{跨页长表格的表题(续)} \\     
		\toprule     
		表头 1 & 表头 2 & 表头 3 & 表头 4 \\     
		\midrule   
	\endhead     
		\bottomrule   
	\endfoot   
	Row 1  & & & \\   
	Row 2  & & & \\   
	Row 3  & & & \\   
	Row 4  & & & \\   
	Row 5  & & & \\   
	Row 6  & & & \\   
	Row 7  & & & \\   
	Row 8  & & & \\   
	Row 9  & & & \\   
	Row 10 & & & \\ 
\end{longtable}

\section{算法}

算法环境可以使用 \pkg{algorithms} 或者 \pkg{algorithm2e} 宏包。

\renewcommand{\algorithmicrequire}{\textbf{输入:}\unskip}
\renewcommand{\algorithmicensure}{\textbf{输出:}\unskip}
\begin{algorithm}   
	\caption{Calculate $y = x^n$}   
	\label{alg1}   
	\small   
	\begin{algorithmic}     
		\REQUIRE $n \geq 0$     
		\ENSURE $y = x^n$
	
	    \STATE $y \leftarrow 1$     
		\STATE $X \leftarrow x$     
		\STATE $N \leftarrow n$
	
	    \WHILE{$N \neq 0$}       
			\IF{$N$ is even}         
				\STATE $X \leftarrow X \times X$         
				\STATE $N \leftarrow N / 2$       
			\ELSE[$N$ is odd]         
				\STATE $y \leftarrow y \times X$         
				\STATE $N \leftarrow N - 1$       
			\ENDIF     
		\ENDWHILE   
	\end{algorithmic} 
\end{algorithm}

\chapter{数学符号和公式}

\section{数学符号}

中文论文的数学符号默认遵循 GB/T 3102.11---1993《物理科学和技术中使用的数学符号》\footnote{原 GB 3102.11---1993,自 2017 年 3 月 23 日起,该标准转为推荐性标准。}。
该标准参照采纳 ISO 31-11:1992 \footnote{目前已更新为 ISO 80000-2:2019。},
但是与 \textbackslash TeX\{\} 默认的美国数学学会(AMS)的符号习惯有所区别。 具体地来说主要有以下差异:
\begin{enumerate}
\item 大写希腊字母默认为斜体,如
\[
\Gamma\Delta\Theta\Lambda\Xi\Pi\Sigma\Upsilon\Phi\Psi\Omega
\]
注意有限增量符号 $\increment$ 固定使用正体,模板提供了 \textbackslash cs\{increment\}
命令。
\item 小于等于号和大于等于号使用倾斜的字形$\le$、$\ge$。
\item 积分号使用正体,比如 $\int$、$\oint$。
\item 行间公式积分号的上下限位于积分号的上下两端,比如
\[
\int_{a}^{b}f(x)\dif x.
\]
行内公式为了版面的美观,统一居右侧,如 $\int_{a}^{b}f(x)\dif x$ 。
\item 偏微分符号 $\partial$ 使用正体。
\item 省略号 \cs{dots} 按照中文的习惯固定居中,比如
\[
1,2,\dots,n\quad1+2+\dots+n.
\]
\item 实部 $\Re$ 和虚部 $\Im$ 的字体使用罗马体。
\end{enumerate}
\par

以上数学符号样式的差异可以在模板中统一设置。 另外国标还有一些与 AMS 不同的符号使用习惯,需要用户在写作时进行处理:
\begin{enumerate}
\item 数学常数和特殊函数名用正体,如
\[
\uppi=3.14\dots;\quad\symup{i}^{2}=-1;\quad\symup{e}=\lim_{n\to\infty}\left(1+\frac{1}{n}\right)^{n}.
\]
\item 微分号使用正体,比如 $\dif y/\dif x$。
\item 向量、矩阵和张量用粗斜体(\cs{symbf}),如$\symbf{x}$、$\symbf{\Sigma}$、$\symbfsf{T}$。
\item 自然对数用 $\ln x$ 不用 $\log x$。
\end{enumerate}
\par

英文论文的数学符号使用 \TeX{} 默认的样式。 如果有必要,也可以通过设置 \verb|math-style| 选择数学符号样式。

关于量和单位推荐使用 \href{http://mirrors.ctan.org/macros/latex/contrib/siunitx/siunitx.pdf}{siunitx}
宏包, 可以方便地处理希腊字母以及数字与单位之间的空白, 比如:

\SI{6.4e6}{m}, 
\SI{9}{\micro\meter}, 
\si{kg.m.s^{-1}}, 
\SIrange{10}{20}{\degreeCelsius}。

\section{数学公式}

数学公式可以使用\env{equation} 和 \env{equation*} 环境。 注意数学公式的引用应前后带括号,建议使用
\cs{eqref} 命令,比如式\ref{eq:example}。
\begin{equation}
\frac{1}{2\uppi\symup{i}}\int_{\gamma}f=\sum_{k=1}^{m}n(\gamma;a_{k})\mathscr{R}(f;a_{k})\label{eq:example}
\end{equation}

注意公式编号的引用应含有圆括号,可以使用 \cs{eqref} 命令。

多行公式尽可能在“=”处对齐,推荐使用\env{align} 环境。
\begin{align}
a & =b+c+d+e\label{eq:}\\
 & =f+g\label{eq:-1}
\end{align}


\section{数学定理}

定理环境的格式可以使用 \pkg{amsthm} 或者 \pkg{ntheorem} 宏包配置。 用户在导言区载入这两者之一后,模板会自动配置
\env{thoerem}、\env{proof} 等环境。

\begin{theorem}[Lindeberg--Lévy 中心极限定理]   
	设随机变量 $X_1, X_2, \dots, X_n$ 独立同分布, 且具有期望 $\mu$ 和有限的方差 $\sigma^2 \ne 0$,   
	记 $\bar{X}_n = \frac{1}{n} \sum_{i+1}^n X_i$,则   
	\begin{equation}     
		\lim_{n \to \infty} P \left(\frac{\sqrt{n} \left( \bar{X}_n - \mu \right)}{\sigma} \le z \right) = \Phi(z),   	\end{equation}   
	其中 $\Phi(z)$ 是标准正态分布的分布函数。 
\end{theorem} 
\begin{proof}   
	Trivial. 
\end{proof}

同时模板还提供了\env{assumption}、\env{definition}、\env{proposition}、\env{lemma}、\env{theorem}、\env{axiom}、\env{corollary}、\env{exercise}、\env{example}、\env{remar}、\env{problem}、\env{conjecture}
这些相关的环境。

\chapter{引用文献的标注}

模板支持 BibTeX 和 BibLaTeX 两种方式处理参考文献。 下文主要介绍 BibTeX 配合 \pkg{natbib}
宏包的主要使用方法。

\section{顺序编码制}

在顺序编码制下,默认的 \cs{cite} 命令同 \cs{citep} 一样,序号置于方括号中, 引文页码会放在括号外。 统一处引用的连续序号会自动用短横线连接。

\sysusetup{   cite-style = super, }
\begin{center}
\begin{tabular}{ll}
\textbackslash cite\{zhangkun1994\} & \quad$\Rightarrow$\quad\cite{zhangkun1994}\tabularnewline
\textbackslash citet\{zhangkun1994\} & \quad$\Rightarrow$\quad\citet{zhangkun1994}\tabularnewline
\textbackslash citep\{zhangkun1994\} & \quad$\Rightarrow$\quad\citep{zhangkun1994}\tabularnewline
\textbackslash cite{[}42{]}\{zhangkun1994\} & \quad$\Rightarrow$\quad\cite[42]{zhangkun1994}\tabularnewline
\textbackslash cite\{zhangkun1994,zhukezhen1973\} & \quad$\Rightarrow$\quad\cite{zhangkun1994,zhukezhen1973}\tabularnewline
\end{tabular}
\par\end{center}

也可以取消上标格式,将数字序号作为文字的一部分。 建议全文统一使用相同的格式。

\sysusetup{   cite-style = inline, }
\begin{center}
\begin{tabular}{ll}
\textbackslash cite\{zhangkun1994\} & \quad$\Rightarrow$\quad\cite{zhangkun1994}\tabularnewline
\textbackslash citet\{zhangkun1994\} & \quad$\Rightarrow$\quad\citet{zhangkun1994}\tabularnewline
\textbackslash citep\{zhangkun1994\} & \quad$\Rightarrow$\quad\citep{zhangkun1994}\tabularnewline
\textbackslash cite{[}42{]}\{zhangkun1994\} & \quad$\Rightarrow$\quad\cite[42]{zhangkun1994}\tabularnewline
\textbackslash cite\{zhangkun1994,zhukezhen1973\} & \quad$\Rightarrow$\quad\cite{zhangkun1994,zhukezhen1973}\tabularnewline
\end{tabular}
\par\end{center}

\section{著者-出版年制}

著者-出版年制下的 \textbackslash cs\{cite\} 跟 \textbackslash cs\{citet\}
一样。

\sysusetup{   cite-style = author-year, }
\begin{center}
\begin{tabular}{ll}
\textbackslash cite\{zhangkun1994\} & \quad$\Rightarrow$\quad\cite{zhangkun1994}\tabularnewline
\textbackslash citet\{zhangkun1994\} & \quad$\Rightarrow$\quad\citet{zhangkun1994}\tabularnewline
\textbackslash citep\{zhangkun1994\} & \quad$\Rightarrow$\quad\citep{zhangkun1994}\tabularnewline
\textbackslash cite{[}42{]}\{zhangkun1994\} & \quad$\Rightarrow$\quad\cite[42]{zhangkun1994}\tabularnewline
\textbackslash cite\{zhangkun1994,zhukezhen1973\} & \quad$\Rightarrow$\quad\cite{zhangkun1994,zhukezhen1973}\tabularnewline
\end{tabular}
\par\end{center}

\vspace{2ex}

\sysusetup{   cite-style = super, }

注意,引文参考文献的每条都要在正文中标注\cite{aaas1883science,abrahams99tex,atkinson1982experimental,baishunong1998zhiwu,biaozhunhua2002tushu,bixon1996dynamics,carlson1981two,chubanzhuanye2004,dizhi1936dizhi,dupont1974bone,fugang2000fengsha,guangxi1993,hanjiren1985lun,huosini1989guwu,jianduju1994,jiangxizhou1980,kusch1975perturbations,mahui1995,mellinger1996laser,merkt1995rotational,oclc2000about,peebles2001probability,salomon1995advanced,scitor2000project,shimizu1983laser,taylor1981study,taylor1983scanning,tushuguan1957tushuguanxue,wangfuzhi1865songlun,weinstein1974pathogenic,who1970factors,xiaoyu2001chubanye,zhangkun1994,zhaoyaodong1998xinshidai,zhengkaiqing1987,zhukezhen1973}

\backmatter{}

\bibliographystyle{sysuthesis-numeric}
\bibliography{ref/refs,ref/appendix}


\appendix{}

\chapter{补充内容}

附录是与论文内容密切相关、但编入正文又影响整篇论文编排的条理和逻辑性的资料,例如某些重要的数据表格、计算程序、统计表等,是论文主体的补充内容,可根据需要设置。

\section{图表示例}

\subsection{图}

附录中的图片示例(图\textasciitilde\ref{fig:appendix-figure})。

\begin{figure}
\begin{centering}
\includegraphics[width=0.6\linewidth]{figures/example-image-a}
\par\end{centering}
\caption{\label{fig:appendix-figure}附录中的图片示例}

\end{figure}


\subsection{表格}

附录中的表格示例(表\textasciitilde\ref{tab:appendix-table})。 

\begin{table}
\caption{\label{tab:appendix-table}附录中的表格示例 }

\begin{centering}
\begin{tabular}{ll}
\hline 
文件名 & 描述\tabularnewline
\hline 
thuthesis.dtx & 模板的源文件,包括文档和注释\tabularnewline
thuthesis.cls & 模板文件\tabularnewline
thuthesis-{*}.bst & BibTeX 参考文献表样式文件\tabularnewline
thuthesis-{*}.bbx & BibLaTeX 参考文献表样式文件\tabularnewline
thuthesis-{*}.cbx & BibLaTeX 引用样式文件\tabularnewline
\hline 
\end{tabular}
\par\end{centering}
\end{table}


\section{数学公式}

附录中的数学公式示例(公式\ref{eq:appendix-equation})。
\begin{equation}
\frac{1}{2\pi i}\int_{\gamma}f=\sum_{k=1}^{m}n\left(\text{\ensuremath{\gamma}};a_{k}\right)\mathscr{R}\left(f;a_{k}\right)\label{eq:appendix-equation}
\end{equation}

\begin{acknowledgements}
衷心感谢导师×××教授和物理系××副教授对本人的精心指导。他们的言传身教将使我终生受益。

在美国麻省理工学院化学系进行九个月的合作研究期间,承蒙 Robert Field 教授热心指导与帮助,不胜感激。

感谢×××××实验室主任×××教授,以及实验室全体老师和同窗们学的热情帮助和支持!

本课题承蒙国家自然科学基金资助,特此致谢。
\end{acknowledgements}
\begin{resume}
\textbf{\section*{个人简历}}

197× 年 ×× 月 ×× 日出生于四川××县。

1992 年 9 月考入××大学化学系××化学专业,1996 年 7 月本科毕业并获得理学学士学位。

1996 年 9 月免试进入xx大学化学系攻读××化学博士至今。 

\section*{在学期间完成的相关学术成果}
\subsection*{学术论文}   
\nobibliography*
\begin{achievements}     
	\item \bibentry{zhangkun1994}
	\item Yang Y, Ren T L, Zhang L T, et al. Miniature microphone with silicon-based ferroelectric thin films[J]. Integrated Ferroelectrics, 2003, 52:229-235.     
	\item 杨轶, 张宁欣, 任天令, 等. 硅基铁电微声学器件中薄膜残余应力的研究[J]. 中国机械工程, 2005, 16(14):1289-1291.     
	\item 杨轶, 张宁欣, 任天令, 等. 集成铁电器件中的关键工艺研究[J]. 仪器仪表学报, 2003, 24(S4):192-193.     
	\item Yang Y, Ren T L, Zhu Y P, et al. PMUTs for handwriting recognition. In press[J]. (已被Integrated Ferroelectrics录用)   \end{achievements}
\subsection*{专利}
\begin{achievements}     
	\item 任天令, 杨轶, 朱一平, 等. 硅基铁电微声学传感器畴极化区域控制和电极连接的方法: 中国, CN1602118A[P]. 2005-03-30.     
	\item Ren T L, Yang Y, Zhu Y P, et al. Piezoelectric micro acoustic sensor based on ferroelectric materials: USA, No.11/215, 102[P]. (美国发明专利申请号.)   
\end{achievements}   
\end{resume}
\begin{committee}
\newcolumntype{C}[1]{@{}>{\centering\arraybackslash}p{#1}}
  \section*{指导小组名单}
  \begin{center}     \begin{tabular}{C{3cm}C{3cm}C{9cm}@{}}       李XX & 教授     & 清华大学 \\       王XX & 副教授   & 清华大学 \\       张XX & 助理教授 & 清华大学 \\     \end{tabular}   \end{center}
  \section*{公开评阅人名单}
  \begin{center}     \begin{tabular}{C{3cm}C{3cm}C{9cm}@{}}       刘XX & 教授   & 清华大学                    \\       陈XX & 副教授 & XXXX大学                    \\       杨XX & 研究员 & 中国XXXX科学院XXXXXXX研究所 \\     \end{tabular}   \end{center}
  \section*{答辩委员会名单}
  \begin{center}     \begin{tabular}{C{2.75cm}C{2.98cm}C{4.63cm}C{4.63cm}@{}}       主席 & 赵XX                  & 教授                    & 清华大学       \\       委员 & 刘XX                  & 教授                    & 清华大学       \\           & \multirow{2}{*}{杨XX} & \multirow{2}{*}{研究员} & 中国XXXX科学院 \\           &                       &                         & XXXXXXX研究所  \\           & 黄XX                  & 教授                    & XXXX大学       \\           & 周XX                  & 副教授                  & XXXX大学       \\       秘书 & 吴XX                  & 助理研究员              & 清华大学       \\     \end{tabular}   \end{center}
\end{committee}
\begin{resolution}
论文提出了……

论文取得的主要创新性成果包括:

1. ……

2. ……

3. ……

论文工作表明作者在×××××具有×××××知识,具有××××能力,论文××××,答辩××××。

答辩委员会表决,(×票/一致)同意通过论文答辩,并建议授予×××(姓名)×××(门类)学博士/硕士学位。 
\end{resolution}

\backcoverpage[scan/scan-backcover.pdf]
\end{document}
